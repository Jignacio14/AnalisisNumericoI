\section{Desarrollo}
\subsection{Euler explícito}
El método de Euler explícito es de la siguiente manera:
\begin{equation}
u_{n+1} = u_n + h f(u_n, t_n)    
\end{equation}
Para el caso de estudio tenemos:
\[
\begin{cases}
    f_1(u_n,v_n,t_n) = v_n \\
    f_2(u_n,v_n,t_n) = \frac{k}{m}(c - u_n) + \frac{\lambda}{m}(c' - v_n)
\end{cases}
\]
Por lo tanto tenemos las ecuaciones:
\[
\begin{cases}
    u_{n+1} = u_n+hv_n \\
    v_{n+1} = v_n + h [\frac{k}{m}(c - u_n) + \frac{\lambda}{m}(c' - v_n)]
\end{cases}
\]
\textcolor{red}{\textbf{SEGUIR}}
\subsection{Euler implícito}
El método de Euler implícito es de la siguiente manera:
\begin{equation}
u_{n+1} = u_n + h f(u_{n+1}, t_{n+1})
\end{equation}
Para el caso de estudio tenemos:
\[
\begin{cases}
    f_1(u_{n+1},v_{n+1},t_{n+1}) = v_{n+1} \\
    f_2(u_{n+1},v_{n+1},t_{n+1}) = \frac{k}{m}(c - u_{n+1}) + \frac{\lambda}{m}(c' - v_{n+1})
\end{cases}
\]
Por lo tanto tenemos las ecuaciones:
\[
\begin{cases}
    u_{n+1} = u_n+hv_n \\
    v_{n+1} = v_n + h [\frac{k}{m}(c - u_n) + \frac{\lambda}{m}(c' - v_n)]
\end{cases}
\]
\textcolor{red}{\textbf{SEGUIR}}
\subsection{Runge Kutta de orden 2}
El método de RK2 es de la siguiente manera:
$
\begin{cases}
    q_1 = h f(u_n, t_n) \\
    q_2 = h f(u_n + q_1, t_{n+1}) \\
    u_{n+1} = u_n + \frac{1}{2} (q_1 + q_2)    
\end{cases}
$

\textcolor{red}{\textbf{ME TRABÉ AYUDA}}

Para el problema planteamos las ecuaciones:
$
\begin{cases}
    q_1 = hv_n \\
    q_2 = hf(u_n+hv_n,t_{n+1}) \\
    u_{n+1} = u_n + \frac{1}{2} (q_1+q_2)    
\end{cases}
$