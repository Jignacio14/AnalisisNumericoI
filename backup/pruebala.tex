%
% Carátula oficial de 75.02 Algoritmos y Programación I, cátedra Cardozo.
%
% Basado en el template realizado por Diego Essaya, disponible en
%                                                         http://lug.fi.uba.ar
% Modificado por Michel Peterson.
% Modificado por Sebastián Santisi.

%
% Acá se define el tamaño de letra principal:
%
\documentclass[12pt]{article}

%
% Título y autor(es):
%
\title{Trabajo Práctico N\b o X}
\author{Apellido1\\Apellido2}

%------------------------- Carga de paquetes ---------------------------
%
% Si no necesitás algún paquete, comentalo.
%

%
% Definición del tamaño de página y los márgenes:
%
\usepackage[a4paper,headheight=16pt,scale={0.7,0.8},hoffset=0.5cm]{geometry}

%
% Vamos a escribir en castellano:
%
\usepackage[spanish]{babel}
\usepackage[utf8]{inputenc}

%
% Si preferís el tipo de letra Helvetica (Arial), descomentá las siguientes
% dos lineas (las fórmulas seguirán estando en Times):
%
%\usepackage{helvet}
%\renewcommand\familydefault{\sfdefault}

%
% El paquete amsmath agrega algunas funcionalidades extra a las fórmulas. 
% Además defino la numeración de las tablas y figuras al estilo "Figura 2.3", 
% en lugar de "Figura 7". (Por lo tanto, aunque no uses fórmulas, si querés
% este tipo de numeración dejá el paquete amsmath descomentado).
%
\usepackage{amsmath}
\usepackage{xcolor}
\numberwithin{equation}{section}
\numberwithin{figure}{section}
\numberwithin{table}{section}
\usepackage{svg}

%
% Para tener cabecera y pie de página con un estilo personalizado:
%
\usepackage{fancyhdr}

%
% Para poner el texto "Figura X" en negrita:
% (Si no tenés el paquete 'caption2', probá con 'caption').
%
\usepackage[hang,bf]{caption}

%
% Para poder usar subfiguras: (al estilo Figura 2.3(b) )
%
%\usepackage{subfigure}

%
% Para poder agregar notas al pie en tablas:
%
%\usepackage{threeparttable}

%------------------------------ graphicx ----------------------------------
%
% Para incluir imágenes, el siguiente código carga el paquete graphicx 
% según se esté generando un archivo dvi o un pdf (con pdflatex). 

% Para generar dvi, descomentá la linea siguiente:
%\usepackage[dvips]{graphicx}

% Para generar pdf, descomentá las dos lineas seguientes:
\usepackage[pdftex]{graphicx}
\pdfcompresslevel=9

%
% Todas las imágenes están en el directorio tp-img:
%
\newcommand{\imgdir}{includes}
\graphicspath{{\imgdir/}}
%
%------------------------------ graphicx ----------------------------------

% Necesitas este paquete si haces los diagrámas de flujo en el prográma Dia 
%\usepackage{tikz}


%------------------------- Inicio del documento ---------------------------

\begin{document}

%
% Hago que en la cabecera de página se muestre a la derecha la sección,
% y en el pie, en número de página a la derecha:
%
\pagestyle{fancy}
\renewcommand{\sectionmark}[1]{\markboth{}{\thesection\ \ #1}}
\lhead{}
\chead{}
\rhead{\rightmark}
\lfoot{}
\cfoot{}
\rfoot{\thepage}

%
% Carátula:
%
\begin{titlepage}

\thispagestyle{empty}

\begin{center}
\includegraphics[scale=0.3]{fiuba}\\
\large{\textsc{Universidad de Buenos Aires}}\\
\large{\textsc{Facultad De Ingeniería}}\\
\small{Año 2023 - 2\textsuperscript{do} Cuatrimestre}
\end{center}

\vfill

\begin{center}
\Large{\underline{\textsc{Análisis Numérico I (75.12 – 95.04)}}}
\end{center}

\vfill

\begin{tabbing}
\hspace{2cm}\=\+TRABAJO PRÁCTICO\\
	TEMA:Resolución numérica de problemas de valores iniciales\\
	FECHA:21 de Noviembre 2023\\% \today\\
\\
	INTEGRANTES:\hspace{-1cm}\=\+\hspace{1cm}\=\hspace{6cm}\=\\
		Aramayo Zambrana, Carolina Luna	\>\> \#106260\\
			\>\footnotesize{$<$caramayo@fi.uba.ar$>$}\\
		Castro Martinez, José Ignacio	\>\> \#106957\\
			\>\footnotesize{$<$nombre2\_apellido2@fi.uba.ar$>$}\\
		Buchanan, Felix Tomas	\>\> \#102665\\
			\>\footnotesize{$<$nombre3\_apellido3@fi.uba.ar$>$}\\
\end{tabbing}

\vfill

\hrule
\vspace{0.2cm}

\noindent\small{Análisis Numérico I (75.12 – 95.04) \hfill Catedra Cavaliere}

\end{titlepage}

%
% Hago que las páginas se comiencen a contar a partir de aquí:
%
\setcounter{page}{1}

%
% Pongo el índice en una página aparte:
%
\tableofcontents
\newpage

%
% Inicio del TP:
%
\section{Introducción}
%\flushleft\includegraphics[scale=0.75]{specification.pdf}
Este informe se propone abordar la resoulusión numérica de problemas de valores iniciales a través de los métodos Euler explícito, Euler implícito y Runge Jutta de orden 2. Se quiere resolver un sistema de suspensión vehicualer, el cual se modela como un sistema oscilatorio amortiguador expresado por una ecuación diferencial de segundo grado.
\section{Análisis}
Para empezar, realizamos un análisis del problema de forma física:
\begin{figure}[h] 
    \centering
    \includesvg{analisis}
    \caption{Análisis del caso de estudio}
    \label{fig:mi_figura}
\end{figure}

\begin{figure}[h] 
    \centering
    \includesvg{circuitoMecanico}
    \caption{Circuito mecánico}
    \label{fig:mi_figura}
\end{figure}
El enunciado nos da una ecuación diferencial que es la siguiente:
\begin{align}
    y'' &= \frac{k}{m}(c - y) + \frac{\lambda}{m}(c' - y')
\end{align}
Esta representa la aceleración vertical de la carrocería. Es decir, oscilador amortiguado que responde a una excitación dada por la variable c.
\begin{itemize}
    \item \( k \) constante elástica del muelle [N/m]
    \item \( \lambda \) constante de amortiguación [Ns/m]
    \item \( c \) cota o elevación del terreno [m]
    \item \( y \) posición de la carrocería [m]
    \item \( c' \) y \( y' \) son derivadas de \( c \) e \( y \) con respecto al tiempo, es decir, velocidades verticales [m/s]
\end{itemize}
Entonces tenemos a \(c(t)\) y \(y(t)\) por hallar.
Como la solución general se solicita en el punto 1, desarrollaremos los métodos de forma genérica y luego se aplicará en el siguiente punto con lo solicitado.
Para ello debemos discretizar la EDO de segundo orden por los métodos propuestos.
Vamos a discretizar tres parámetros:
\subsection{Discretizar la EDO}
Como es una EDO de segundo orden se debe definir dos nuevas variables
\[
\begin{cases}
    y(t) \approx y(t_n) \approx u_n \\
    y'(t) \approx y'(t_n) \approx u'_n = v_n 
\end{cases}
\]
De esta forma pasamos de una EDO de segundo orden a una EDO lineal
\[
\begin{cases}
    u' = f_1(u,v,t) \\
    v' = f_2(u,v,t)
\end{cases}
\]
\pagebreak
\section{Desarrollo}
\subsection{Euler explícito}
El método de Euler explícito es de la siguiente manera:
\begin{equation}
u_{n+1} = u_n + h f(u_n, t_n)    
\end{equation}
Para el caso de estudio tenemos:
\[
\begin{cases}
    f_1(u_n,v_n,t_n) = v_n \\
    f_2(u_n,v_n,t_n) = \frac{k}{m}(c - u_n) + \frac{\lambda}{m}(c' - v_n)
\end{cases}
\]
Por lo tanto tenemos las ecuaciones:
\[
\begin{cases}
    u_{n+1} = u_n+hv_n \\
    v_{n+1} = v_n + h [\frac{k}{m}(c - u_n) + \frac{\lambda}{m}(c' - v_n)]
\end{cases}
\]
\textcolor{red}{\textbf{SEGUIR}}
\subsection{Euler implícito}
El método de Euler implícito es de la siguiente manera:
\begin{equation}
u_{n+1} = u_n + h f(u_{n+1}, t_{n+1})
\end{equation}
Para el caso de estudio tenemos:
\[
\begin{cases}
    f_1(u_{n+1},v_{n+1},t_{n+1}) = v_{n+1} \\
    f_2(u_{n+1},v_{n+1},t_{n+1}) = \frac{k}{m}(c - u_{n+1}) + \frac{\lambda}{m}(c' - v_{n+1})
\end{cases}
\]
Por lo tanto tenemos las ecuaciones:
\[
\begin{cases}
    u_{n+1} = u_n+hv_n \\
    v_{n+1} = v_n + h [\frac{k}{m}(c - u_n) + \frac{\lambda}{m}(c' - v_n)]
\end{cases}
\]
\textcolor{red}{\textbf{SEGUIR}}
\subsection{Runge Kutta de orden 2}
El método de RK2 es de la siguiente manera:
\begin{cases}
q_1 = hf(u_n,t_n) \\
q_2 = hf(u_n+q_1,t_{n+1}) \\
u_{n+1} = u_n + \frac{1}{2} (q_1+q_2)    
\end{cases}

\textcolor{red}{\textbf{ME TRABÉ AYUDA}}

Para el problema planteamos las ecuaciones:
\begin{cases}
q_1 = hv_n \\
q_2 = hf(u_n+hv_n,t_{n+1}) \\
u_{n+1} = u_n + \frac{1}{2} (q_1+q_2)    
\end{cases}
\pagebreak
\section{Consideraciónes y Estrategias}
\pagebreak
\section{Resultados de ejecución}
\pagebreak
\section{Conclusión}
\pagebreak

\end{document}

